\documentclass{thesis}
\usepackage{lipsum}

\begin{document}

\newgeometry{top=2cm,bottom=2.5cm,right=3cm}
\begin{titlepage}

    \begin{tikzpicture}[remember picture, overlay]
        \draw[line width=1.2mm] ($(current page.north west) + (3cm, -3cm)$) rectangle ($(current page.south east) + (-3cm,3cm)$);
        \draw[ultra thick] ($(current page.north west) + (3.15cm,-3.15cm)$) rectangle ($(current page.south east) + (-3.15cm,3.15cm)$);
    \end{tikzpicture}

    \begin{center}
    \vspace{0.3cm}
        {\bf\MakeUppercase{Đại học quốc gia Hà Nội}}         \\

        {\bf\MakeUppercase{Tên trường}} \\
        \vspace{1cm}
        \includegraphics[width=100pt]{images/uet.png} \\
        \vspace{1cm}
        {\fontsize{14pt}{14pt}\textbf{Họ và tên}} \\
        \vspace{1cm}
        {\fontsize{18pt}{18pt}\textbf{Tên đề tài}}
        \vfill{}
        \vspace{1cm}
        {\fontsize{14pt}{14pt}\textbf{KHÓA LUẬN TỐT NGHIỆP ĐẠI HỌC HỆ CHÍNH QUY}} \\
        {\fontsize{14pt}{14pt}\textbf{Ngành: }\ldots} \\
        \vfill{}

        {\fontsize{12pt}{12pt}\bf{Hà Nội: 2021}}
        \vspace{0.8cm}
    \end{center}
\end{titlepage}

\begin{titlepage}
    \begin{tikzpicture}[remember picture, overlay]
        \draw[line width=1.2mm] ($(current page.north west) + (3cm, -3cm)$) rectangle ($(current page.south east) + (-3cm,3cm)$);
        \draw[ultra thick] ($(current page.north west) + (3.15cm,-3.15cm)$) rectangle ($(current page.south east) + (-3.15cm,3.15cm)$);
    \end{tikzpicture}

    \begin{center}
        \vspace{0.3cm}
        {\bf\MakeUppercase{Đại học quốc gia Hà Nội}}
        \\
        {\bf\MakeUppercase{Tên trường}} \\
        \vspace{2cm}
        {\fontsize{14pt}{14pt}\bf{Họ và tên}} \\
        \vspace{2cm}
        {\fontsize{18pt}{18pt}\textbf{Tên đề tài}} \\
        \vspace{3cm}
        {\fontsize{14pt}{14pt}\bf{KHÓA LUẬN TỐT NGHIỆP ĐẠI HỌC HỆ CHÍNH QUY}} \\

        {\fontsize{14pt}{14pt}\bf{Ngành: }\ldots} \\
    \end{center}
    \vspace{3cm}
    \begin{flushleft}
        {\fontsize{14pt}{14pt} \hspace{0.5cm} \textbf{Cán bộ hướng dẫn:} \fontsize{14pt}{14pt}\textbf{Học vị. Họ và tên}}
    \end{flushleft}
    \vfill
    \begin{center}
            {\fontsize{12pt}{12pt}\bf{Hà Nội: 2021}}
    \end{center}
    \vspace{0.8cm}
\end{titlepage}

\restoregeometry
\pagebreak

\begin{titlepage}
    \begin{center}
        {\bf TÓM TẮT}
    \end{center}
    \par \textbf{Tóm tắt:} {\fontsize{12pt}{12pt} nội dung}
    % in nghiêng, cách nhau bởi dấu phẩy
    \par \bfit{Từ khóa:} 
\end{titlepage}

\pagebreak


\begin{titlepage}
    \begin{center}
        \bf{LỜI CẢM ƠN}
    \end{center}

    \par Tôi xin chân thành gửi lời cảm ơn đến các thầy cô của Trường \ldots $-$ Đại học Quốc Gia Hà Nội đã tận tình hướng dẫn, giảng dạy và động viên trong suốt quá trình học tập, rèn luyện tại trường để tôi có được những kiến thức và suy nghĩ như ngày hôm nay.
    
    \par Tôi xin chân thành gửi lời cảm ơn sâu sắc đến \ldots đã tận hình hướng dẫn, chỉ bảo tôi xuyên suốt quá trình thực hiện khóa luận tốt nghiệp này.
    
    \par Cuối cùng là lời cảm ơn đến gia đình, bạn bè, các em khóa dưới $-$ những người đã luôn sát cánh bên tôi những lúc khó khăn, luôn hỏi thăm và ủng hộ tôi để tôi có thể hoàn thành khóa luận tốt nghiệp này.
    
    \par Một lần nữa, tôi xin chân thành cảm ơn.
\end{titlepage}
\pagebreak
    
\begin{titlepage}
    \begin{center}
        \bf{LỜI CAM ĐOAN}
    \end{center}
    
    \par Tôi xin cam đoan rằng mọi kết quả trình bày trong khóa luận đều do tôi thực hiện dưới sự hướng dẫn của \ldots Tất cả những tài liệu tham khảo, nghiên cứu liên quan sử dụng trong khóa luận tốt nghiệp đều được nêu rõ trong phần tài liệu tham khảo. Khóa luận tốt nghiệp không sao chép tài liệu, công trình nghiên cứu từ người khác mà là thành quả của bản thân. 
    
    \begin{flushright}
        \begin{tabular}{@{}c@{}}
        Hà Nội, \MakeLowercase{\today}\\
        Sinh viên \\
        \\
        \\
        \\
        Họ và tên
        \end{tabular}
    \end{flushright}
\end{titlepage}

\pagebreak



\cleardoublepage
\pagenumbering{gobble}

\tableofcontents
\pagebreak

\listoffigures
\pagebreak

\cleardoublepage
\pagenumbering{arabic}

\addtocontents{toc}{\protect\thispagestyle{empty}}

\section{Section 1}

\subsection{Subsection 1.1}

\lipsum[10-15]

\subsection{Subsection 1.2}

\lipsum[11-14]

\newpage % tạo section mới là phải sang trang mới
\section{Section 2}

\subsection{Subsection 2.1}

\lipsum[1-8]

\subsection{Subsection 2.2}

\lipsum[3-11]

\newpage % tạo section mới là phải sang trang mới
\section{Section 3}

\subsection{Subsection 3.1}

\lipsum[15-21]

\subsection{Subsection 3.2}

\lipsum[13-17]


\pagebreak
\bibliographystyle{acm}
\bibliography{references}

\end{document}
